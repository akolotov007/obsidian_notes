\documentclass{article}
\usepackage{amsmath, amssymb}
\usepackage{geometry}
\usepackage[utf8]{inputenc}  % Allows input of Unicode characters
\geometry{margin=1in}

\title{HW 5}
\date{3/8/25}
\author{Anthony Kolotv}

\begin{document}
	
	\maketitle


	Suppose that $a$ and $b$ are integers, $a \equiv 13 \pmod{19}$, and $b \equiv 5 \pmod{19}$. Find the integer $c$ such that $0 \leq c \leq 18$ (i.e., $c \in \mathbb{Z}_{19}$) satisfying:
	\begin{enumerate}
		\item $c \equiv 13a \pmod{19}$
		\subitem 13 $\times$ 13 = 169/19 = 8.89 (round down)
		\subitem 169 - (8 $\times$ 19) = 169 - 152 = 17 = c
		\item $c \equiv 8b \pmod{19}$ 
		\subitem 8 $\times$ 5 = 40 (mod 19), 40/19 = R2
		\subitem 40 - 2 $\times$ 19 = 40 - 38 = 2 = c
		\item $c \equiv a - 2b \pmod{19}$
		\subitem 13 - 2(5) = 3 (mod 19) = c
		\item $c \equiv 7a + 3b \pmod{19}$
		\subitem 7 $\times$ 13 + 3 $\times$ 5 = 91+15 = 106 
		\subitem 106/19 = 5.57, 106 - (5 $\times$ 19) = 11 = c
		\item $c \equiv 2a^2 + 3b^2 \pmod{19}$
		\subitem 2($13^2$) + 3($5^2$) = 338 + 75 = 413 
		\subitem 413 / 19 = 21.74, 413 - (21 $\times$ 19) = 14 = c
		\item $c \equiv a^3 + 4b^3 \pmod{19}$
		\subitem $13^3$ + 4 ($5^3$) = 2697
		\subitem 2697/19 = 141.9, 2697 - (141 $\times$ 19) = 18 = c
	\end{enumerate}
	

	Evaluate the following congruence's:
	\begin{enumerate}
		\item $-101 \pmod{11}$
		\subitem -101 / 11 = 9.18, 101 - (9 $\times$ 11) = 2
		\subitem 101 $\equiv$ 2 (mod 11) $|$ since we started with -101, we take the neg:
		\subitem -101 $\equiv$ -2 (mod 11) $|$ -2+11 = 9 $|$ get positive remainder
		\subitem -101 $\equiv$ 9 (mod 11)
		\item $(-3)^{100} \pmod{24}$
		\subitem -3 + 24 = 21 $|$ positive equivalent 
		\subitem $21^{100} \pmod{24}$
		\subitem $21^2$ = 441, 441/24 = 18, 441 -(18 $\times$ 24) = 9
		\subitem $21^3$ = 9,261, 9261/24 = 385, 9261 -(365 $\times$ 24) = 21
		\subitem  if you continue - there will be  a repeating pattern with even numbers having remainder 9 and odd numbers remainder 21 
		\subitem since 2 is even and 100 is even, $\therefore$ $-3^{100} \pmod{24} = 9 $ 
		\item $(185 \pmod{23})^2 \pmod{31}$
		\subitem 185 / 23 = 8, 185 - (8 $\times$ 23) = 185-184=1
		\subitem $1^2$ = 1, 1 $\pmod{31}$ = 1
	\end{enumerate}
	

	Convert each of the following binary expansions into decimal, octal, and hexadecimal expansions. Show your steps.
	\begin{enumerate}
		\item $(10 0000 0001)_2$
		\subitem Decimal = $(1\times2^9) + (1\times2^0) = (513)_{10} $
		 \subitem Octal = 1 000 000 001 = $(1001)_8$
		 \subitem Hexadecimal =  [00]10 0000 0001 = $(201)_{16}$
		
		\item $(110 1001 0001 0000)_2$
		\subitem Decimal = $(1\times2^{14}) + (1\times2^{13}) + (1\times2^{11}) +(1\times2^{8}) + (1\times2^{4}) = (26896)_{10} $
		\subitem Octal = 110 100 100 010 000 = $64420_{8}$
		\subitem Hexadecimal = 110 1001 0001 0000 = $6910_{16}$
	\end{enumerate}
	

	Convert the decimal numbers 4077 and 6643 into binary, octal, and hexadecimal expansions. Show your steps.\\
	6643
	\begin{enumerate}
		\item Divide by 2 until you reach 0, any remainders during division process are counted as bits.
		\\ Binary =  6643/2 = 3321 (1), 3321/2 = 1660 (1), 1660/2 = 830 (1), 830/2 = 415 (1), 415/2 = 207 (1), 130/2 = 51 (1), 51/2 = 25 (1), 25 / 2 = 12 (1), 12/2 = 6 (0), 6/2 = 3 (0) 
		\subitem 1100111110011
		\item Divide by 8 until you reach 0, remainders are counted in reverse order.
		\\ Octal = 6643/8 = 830 R3, 830/8 = 103 R6, 103/8 = 12 R7, 12/8 = 1 R4, 1/8 = 0 R1
		\subitem 14763 
		\item Divide by 16 until you reach 0, remainders are counted in reverse order (use modulus). 
		\\ Hexadecimal = 6643/16 = 415 R3, 415/16 = 25 R15, 25/16 = 1 R9 1/16 = 0 R1
		\subitem  = 3 15 9 1 $\rightarrow$ 1 9 15 3 $\rightarrow$ 19F3
	\end{enumerate}
	4077
	\begin{enumerate}
		
		\item Binary = 4077/2 = 2038 (1), 2038/2 = 1019 (0), 1019/2 = 509 (1), 509/2 = 254 (1), 254/2= 127 (0), 127/2 = 63 (1), 63/2 = 31 (1), 31/2=15 (1), 15/2=7 (1), 7/2 = 3 (1), 3/2 = 1 (1)
		\subitem 111111101101
		\item Octal = 4077/8 = 509 R5, 509/8=63 R5, 63/8=7 R7, 7/8 = 0 R7
		\subitem 7755
		\item Hexadecimal = 4077/16 = 254 R 13, 254/16 = 15 R14,  15/16 = 0 R 15
		\subitem 13 14 15 $\rightarrow$ 15 14 13 $\rightarrow$ FED
	\end{enumerate}
		
 	 Use the Euclidean algorithm to find:
	\begin{enumerate}
		\item $\gcd(123, 277)$
		\subitem 277/123 = 2 R31 $| \gcd(123, 31)$
		\subitem 123/31 = 3 $and$ 123 mod 31 = R30 $| \gcd(31,30)$
		\subitem 31/30 = 1 $and$ 31 mod 30 = R1 $| \gcd(30,1)$
		\subitem 30/1 = 30 $and$ 31 mod 1 = R0 $| \gcd(1,0)$
		\subitem =1
		\item $\gcd(1529, 14038)$
		\subitem 14038 / 1529 = 9 R277 , 1529 / 277 = 5 R144, 277/144 = 1 R133, 144/133 = 1 R11, 133/11 = 12 R1, 11/1 = 1 R0 
		\subitem =1
		\item $\gcd(12345, 54321)$
		\subitem 54321 / 12345 = 4 R4941, 12345/4941 = 2 R2463, 4941/2463 = 2 R15, 2463/15 = 164 R3, 15/3= 5 R0 
		\subitem  = 3
		\item $\gcd(9888, 6060)$
		\subitem 12
	\end{enumerate}
	

	Prove that $\sqrt{7}$ is irrational.
\begin{itemize}
	\item Assume that \( \sqrt{7} \) is rational. That is, assume \( \sqrt{7} = \frac{a}{b} \) where \( a \) and \( b \) are integers and \( \frac{a}{b} \) is in simplest form, meaning \( a \) and \( b \) have no common factors other than 1.
	
	\item Squaring both sides, we get:
	\[
	7 = \frac{a^2}{b^2}
	\]
	
	\item Multiplying both sides by \( b^2 \), we obtain:
	\[
	7b^2 = a^2
	\]
	
	\item Since \( a^2 \) is divisible by 7, it follows that \( a \) must also be divisible by 7. Therefore, we can write \( a = 7k \) for some integer \( k \).
	
	\item Substituting \( a = 7k \) into the equation \( 7b^2 = a^2 \), we get:
	\[
	7b^2 = (7k)^2
	\]
	
	\item Simplifying the right-hand side:
	\[
	7b^2 = 49k^2
	\]
	
	\item Dividing both sides by 7:
	\[
	b^2 = 7k^2
	\]
	
	\item This shows that \( b^2 \) is also divisible by 7, and therefore \( b \) must also be divisible by 7.
	
	\item However, we initially assumed that \( a \) and \( b \) had no common factors other than 1. But we have just shown that both \( a \) and \( b \) are divisible by 7, which is a contradiction.
	
	\item Therefore, our assumption that \( \sqrt{7} \) is rational must be false.
	
\end{itemize}

Thus, \( \sqrt{7} \) is irrational.

	\quad
	
	Prove that if $p_1, p_2, \dots, p_n$ are distinct prime numbers with $p_1 = 2$ and $n > 1$, then $p_1 p_2 \cdots p_n + 1$ can be written in the form $4k + 3$ for some integer $k$.
	\subitem  all prime numbers greater than 2 are odd \\ 
	since $p_1$ = 2 and $n>1$ the product $p_1, p_2, \dots, p_n$ can be written as:\\
	$2 \times p_2, \dots, p_n | $ from $p_2 \cdots$ all primes are odd, so their product is odd\\ 
	let $p_2, \dots, p_n = x$ where x is an odd integer, so $2x$.
	Since $x$ is an odd integer, it can be expressed as $(2k+1)$ for some integer $k$:\\
	Thus, $2(2k+1) = 4k+2$ \\ 
	The expression $p_1, p_2, \dots, p_n + 1$ can be written as: $(4k+2)+ 1 = 4k + 3$ \\
	
	
	
	Prove the following statements by mathematical induction:
	\begin{enumerate}
		\item $\sum_{i=1}^{n} i^2 = \frac{n(n+1)(2n+1)}{6}$
		\subitem
		\item $\sum_{i=1}^{n} i^3 = \frac{n^2(n+1)^2}{4} = \left(\sum_{i=1}^{n} i\right)^2$
			\subitem Assume n=1: 
			\subitem$\sum_{i=1}^{1} i^3 = \frac{1^2(1+1)^2}{4} = 1| $ Now we need to prove $n = k+1$ 
		 	\subitem $\sum_{i=1}^{k+1} i^3 = \sum_{i=1}^{k} i^3 + (k+1)^3 | $ using induction hypothesis..
		  	\subitem $\frac{k^2(k+1)^2}{4} + (k+1)^3 | $ Factoring $(k+1)$ ... 
		  	\subitem $\frac{k^2(k+1)^2 + 4(k+1)^3}{4} | $ Factoring $(k+1)^2$ out... 
		  	\subitem $\frac{(k+1)^2(k^2+4(k+1))}{4} = \frac{k+1^2(k+2)^2}{4} | $ This matches the original formula for $n=k+1$  		 
		\item $\sum_{i=1}^{n} i(i!) = (n+1)! - 1$
		\subitem Assume n=1:
		\subitem $\sum_{i=1}^{1} i(i!) = (1+1)! - 1 = 1 | $ Assume the formula holds for n=k+1
		\subitem   $\sum_{i=1}^{k+1} i(i!)  = \sum_{i=1}^{k}i(i!)+(k+1)(k+1)! | $ Using induction hypothesis...
		\subitem $(k+1)!-1 + (k+1)(k+1)! |$ Factor $(k+1)!$ 
		\subitem $(k+1)!(1+(k+1))-1 = (k+2)!-1$ 
		
		
		\item $\prod_{i=0}^{n} \left(\frac{1}{2i+1} \cdot \frac{1}{2i+2}\right) = \frac{1}{(2n+2)!}$
		\subitem assume n=0 (bc i=0, lowest valid case):
		\subitem $\prod_{i=0}^{0} \left(\frac{1}{2(0)+1} \cdot \frac{1}{2(0)+2}\right) =$ $\frac{1}{(2(0)+2)!} = \frac{1}{2} | $ Prove that the formula holds for n=k+1 
		\subitem $\prod_{i=0}^{k+1} \left(\frac{1}{2i+1} \cdot \frac{1}{2i+2}\right) = \frac{1}{(2(k+1)+2)!} | $ Expanding the right side...
	 	\subitem $\frac{1}{(2(k+1)+2)(2(k+1)+1)} = \frac{1}{(2k+3)(2k+4)} | $ Simplifying... 
	 	\subitem $\frac{1}{(2n+2)!} \cdot \frac{1}{(2k+3)(2k+4)}  = \frac{1}{(2k+2)!\cdot(2k+3)(2k+4)} | $
	 	\subitem Notice that $(2k+2)! \cdot (2k+3)(2k+4) = (2k+4)!$ because $(2k+4)!$ includes all the terms of $(2k+2)!$ plus the next two terms, $(2k+3)$ and $(2k+4) | $ Therefore...
	 	\subitem $\prod_{i=0}^{k+1} \left(\frac{1}{2i+1} \cdot \frac{1}{2i+2}\right) = \frac{1}{(2k+4)!}$
	\end{enumerate}
	

	Suppose that we want to prove that:
	\[ \frac{1}{2} \cdot \frac{3}{4} \cdots \frac{2n-1}{2n} < \frac{1}{\sqrt{3n}} \]
	for all positive integers $n$.
	
	\begin{enumerate}
		\item Show that the basis step works, but the inductive step fails.
		\subitem for n = 1:
		\subitem $\frac{1}{2} < \frac{1}{\sqrt{3}} \approx 0.577 = 0.5 < 0.577 |$ base case holds for n=1
		\subitem Prove the formula holds for n = k+1:
		\subitem $\prod_{i=0}^{k+1} \frac{2i-1}{2i} = (\prod_{i=0}^{k} \frac{2i-1}{2i}) \cdot \frac{2k+1}{2k+2} | $ we add $\frac{2k+1}{2k+2}$ to help prove that the inequality holds for k+1.
		\subitem $(\prod_{i=0}^{k} \frac{2i-1}{2i}) \cdot \frac{2k+1}{2k+2} < \frac{1}{\sqrt{3k}} \cdot \frac{2k+1}{2k+2} | $ substituting back into the equation
		\subitem  Now we need to show $\frac{1}{\sqrt{3k}} \cdot \frac{2k+1}{2k+2} < \frac{1}{\sqrt{3(k+1)}} | $ Re-write inequality...
		\subitem  $\frac{2k+1}{2k+2} < \frac{\sqrt{3}}{\sqrt{3(k+1)}} | $ Simplify the right side...
		\subitem $\sqrt{\frac{k}{k+1}} | $ Plugging back into the inequality ...
		\subitem $\frac{2k+1}{2k+2} < \sqrt{\frac{k}{k+1}} | $ Now prove inequality
		\subitem $(\frac{2k+1}{2k+2})^2 < \frac{k}{k+1} | $ Expanding the left side...	
		\subitem $\frac{4k^2+4k+1}{4k^2+8k+4} < \frac{k}{k+1} | $ cross multiply to clear fractions
		\subitem $(4k^2+4k+1)(k+1)<4k^2+8k+4 | $ Simplify...
		\subitem $4k^3+4k^2-3k-3< 0 | $ Analyze...
		\subitem for k=1, $4^3+4^2-3-3 = 64+16-3-3 = 74 \not< 0$
		\subitem Therefore, the induction step fails
		
		\item Show that mathematical induction can be used to prove the stronger inequality:
		\[ \frac{1}{2} \cdot \frac{3}{4} \cdots \frac{2n-1}{2n} < \frac{1}{\sqrt{3n+1}} \]
		for all integers $n > 1$, which, together with verification for $n=1$, establishes the weaker inequality.
		\subitem prove that formula holds for n = k+1:
		\subitem $\prod_{i=1}^{k+1} \frac{2i-1}{2i} < \frac{1}{\sqrt{(3k+1)+1}} | $ We can express the left side as:
		\subitem  $(\prod_{i=1}^{k} \frac{2i-1}{2i}) \cdot \frac{2k+1}{2k+2} | $ showing the full equation...
		\subitem  $(\prod_{i=1}^{k} \frac{2i-1}{2i}) \cdot \frac{2k+1}{2k+2} < \frac{1}{\sqrt{3k+1}} \cdot \frac{2k+1}{2k+2} | $ Now prove that $\frac{1}{\sqrt{3k+1}} \cdot \frac{2k+1}{2k+2} < \frac{1}{\sqrt{(3k+1)+1}} $ ...
		\subitem $\frac{1}{\sqrt{(3k+1)+1}} = \frac{1}{\sqrt{3k+4}}$, 
		\subitem $\frac{1}{\sqrt{3k+1}} \cdot \frac{2k+1}{2k+2} < \frac{1}{\sqrt{3k+4}}  | $ Multiply both sides by $\sqrt{3k+1} \text{ and } \sqrt{3k+4}$ 
		\subitem $\frac{2k+1}{2k+2} < \frac{\sqrt{3k+1}}{\sqrt{3k+4}} | $ Square both sides and expand
		\subitem $\frac{4k^2+4k+1}{4k^2+4k+8} < \frac{\sqrt{3k+1}}{\sqrt{3k+4}} | $ Multiply by both denominators...
		\subitem  \(12k^{3}+28k^{2}+19k+4<12k^{3}+28k^{2}+20k+4 | \) Subtract \(12k^{3}+28k^{2}+19k+4\) from both sides...
		\subitem $0<k$
	\end{enumerate}
	

	An integer sequence $\{a_0, a_1, \dots\}$ is given by $a_0 = 0$ and $a_k = 2a_{k-1} + 1$ for every $k \geq 1$.
	\begin{enumerate}
		\item Calculate by hand the first 5 terms of $a_k$.
		\subitem $a_0 =  0$
		\subitem $a_1 = 2a_{1-1} + 1 = 2(0)+1 = 1$
		\subitem $a_2 = 2a_{2-1} + 1 = 2(1)+1 = 3$
		\subitem $a_3 = 2a_{3-1} + 1 = 2(3)+1 = 7$
		\subitem $a_4 = 2a_{4-1} + 1 = 2(7)+1 = 15$
		
		\item Guess a simple formula for $a_k$.
		\subitem it looks exponential, and each previous term can be multiplied by 2 and adding 1. It could be re-written as $2^k-1$
		\item Prove your conjecture from part (b) by mathematical induction.
		\subitem $a_k = 2^k-1$ for k=0 we have : $a_0 = 0$ (given)
		\subitem the formula gives $2^0 -1 = 1-1 = 0$ so base case holds
		\subitem now prove that formula holds for $k = n+1$:
		\subitem $a^{n+1} = 2^{n+1} + 1$
		\subitem from the relation, we know that: $a_{n+1} = 2a_n+1$, 
		\subitem by inductive hypothesis, $a_n = 2^n-1$, so:
		\subitem $a_{n+1} = 2(2^n-1)+1$, simplifying the right side...
		\subitem $a_{n+1} = 2\cdot 2^n-1+1 = 2^{n+1}-1$ 
		\subitem Thus, the formula holds for $k=n+1$
	\end{enumerate}
	
\end{document}
