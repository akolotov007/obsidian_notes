\documentclass[english,12pt,legalpaper]{article}
\usepackage[T1]{fontenc}
\usepackage{mathtools}
\usepackage{amssymb}
\usepackage{amsthm}
\usepackage{babel}

\title{HW 1}
\author{Anthony Kolotov}
\begin{document}
	\maketitle


\begin{enumerate}
	\item 
		\begin{flushleft}
			(a) Is 0 = {0}? \\
			\begin{itemize}
				\item No, because 0 is a symbolic placeholder for the number, whereas \{0\} is a set that contains an element 0.
			\end{itemize}
			(b) Is \{0\} = $\emptyset$? \\
			\begin{itemize}
				\item No, since the set contains an element, and it does not equal nothingness 	
			\end{itemize}
			(c) How many elements are there in the set \{2, 3, 5, 7, 11, 2, 3, 5, 13\}? \\
			\begin{itemize}
				\item 6 elements 
			\end{itemize}
			(d) How many elements are there in the set \{1, \{1\}, \{1, \{1\}\}\}?\\
			\begin{itemize}
				\item 3 elements
			\end{itemize}
		\end{flushleft}
		
	\item Which of the following sets are equal (recall that we use $\mathbb{Z^+}$ to denote the set of non-negative integers,
	and $\mathbb{Z^{++}}$ to denote the set of positive integers)? Be careful of the difference between discrete sets (finite sets) and infinite sets.

		\begin{flushleft}
	$A = \{0, 1, 2, 3\}$ \\
	$B = \{1, 2, 3\}$ \\ 
	$C = \{x \in  \mathbb{R} | 1 \leq x \leq 3\}$ \\
	$D = \{x \in  \mathbb{R} | 1 < x < 3\}$ \\
	$E = \{x \in  \mathbb{R}^{+} | 1 < x^2 < 9\}$ \\
	$F = \{x \in \mathbb{Z}^{+} | -3 \leq x \leq 3\}$ \\	
	$G = \{x \in \mathbb{Z}^{++} | -1 < x < 4\}$
		\end{flushleft}
		
	\begin{itemize}
		\item A is a finite set of \{0,1,2,3\}
		\item B is a finite set of \{1,2,3\}
		\item C is a continuous set $[1,3]$, 
		\item D is a continuous set (1,3)
		\item E, simplifying the equation, we get two conditions:
		\\ x>1 (x can be restricted to positive values)
		\\ x<3
		\\ (1,3)
		\item F, x in set of non-negative integers,
		\\ \{1,2,3\}
		\item G, x in set of positive integers, 
		\\ \{1,2,3\}
		\\
		\item  B = F = G = \{1,2,3\} 
		\item D = E = (1,3)
	\end{itemize}
	
	
	\item Let $A = \{a, b, c, d\}, B = \{a, b, f \}, \text{ and } C = \{b, f\}.$ Answer each of the following questions. Give reasons for your answers. \\
	(a) Is $B \subseteq A$? 
	\begin{itemize}
		\item No, because $B$ contains an element that is not in $A$ which is $f$
	\end{itemize}
	(b) Is $C \subseteq A$?  
	\begin{itemize}
		\item No, because $C$ contains an element that is not in $A$ which is $f$
	\end{itemize}
	(c) Is $C \subseteq C$?
		\begin{itemize}
			\item Yes, because a set can be a \textit{subset} of itself 
		\end{itemize}
	(d) Is $C$ a proper subset of $B$?
	
	\begin{itemize}
		\item Yes, because $C$ contains  elements that $B$ has, \textit{and} $B$ has an element which isn't in $C$ ($a$).
	\end{itemize}
	\begin{flushright}
	  I do have a question regarding the difference between $\subset$ and $\subseteq$. As I understand it, $\subset$ is a subset, and $\subseteq$ is a proper subset. If you could please correct in my understanding, professor. \\	
	\end{flushright}
	
	
	\item
	Let	$A = \{x,y,z\}$ and $B = \{a,b\}$. Use the set-roster notation to write each of the following sets, and indicate the number of elements that are in each set:
	\begin{flushleft}
		(a) $A \times B$ \\
		\begin{itemize}
			\item \{(x,a),(x,b),(y,a),(y,b),(z,a),(z,b)\}
			\\ 6 elements
		\end{itemize}
		(b) $B \times A$ \\ 
		\begin{itemize}
			\item \{(a,x),(a,y), (a,z),(b,x),(b,y),(b,z)\}
			\\ 6 elements
		\end{itemize}
		(c) $A \times A$ \\
		\begin{itemize}
			\item \{(x,x),(x,y),(x,z),(y,x),(y,y),(y,z),(z,x),(z,y),(z,z)\}
			\\ 9 elements
		\end{itemize}
		(d) $B \times B$
		\begin{itemize}
			\item \{(a,a),(a,b),(b,a),(b,b)\}
			\\ 4 elements
		\end{itemize}
	\end{flushleft}
	
	
	\item
	Let $A = \{2,3,5\}$ and $B = \{6,8,10\}$ and define the relation $\textit{R}$ from $A$ to $B$ as follows: for all $(x,y) \in A \times B, (x,y) \in \textit{R}$ if and only if $\frac{y}{x}$ is a \textit{integer}.
	\begin{flushleft}
	(a) Write $\textit{R}$ as a set of ordered pairs 
	\begin{itemize}
		\item 
		6/2 = 3 \\ 
		8/2 = 4 \\
		10/2 = 5 \\
		6/3 = 2 \\
		8/3 = 2.667 \\
		10/3 = 3.33 \\
		6/5 = 1.2\\ 
		8/5 = 1.6 \\
		10/5 = 2 \\ 
		
		\item \textit{R} = \{(2,6),(2,8),(2,10),(3,6),(5,10)\}
	\end{itemize}
	(b) Write the domain and co-domain of $\textit{R}$ \\
	\begin{itemize}
		\item Domain = \{2,3,5\} \\
		Co-domain  = \{6,8,10\} \\
	\end{itemize}
	(c) Draw an arrow Diagram for $\textit{R}$
		
		\begin{tabular}{c|c|c}
			\textbf{X} &  & \textbf{Y} \\ 
			\hline
			2 &  $\rightarrow$ & 6,8,10 \\ 
			3 & $\rightarrow$ & 6\\
			5 & $\rightarrow$ & 10 \\
			
		\end{tabular}
	
	\end{flushleft}
	
	
	\item Can you modify the domain of the relation $R$ in the previous question to turn $R$ into a function? If so, how?
	\begin{itemize}
		\item In order to make $R$ into a function, the mapping of $x$ would have to be restricted to a single y value. In this case $x=2$ maps to multiple $y$ values.You would need to make that mapping of $x$ to a single $y$ value. 
	\end{itemize}
	
	\item Let $A = \{0,2\}$ and $B = \{1,3,5\}$ and define the relations $U,V,W$ from $A$ and $B$ as follows:
	\begin{center}
		$(x,y) \in U$ if and only if $4 < x+y < 6$ \\ 
		$(x,y) \in V$ if and only if $y - 1 = \frac{x}{2}$ \\ 
		$W = \{(0,3), (2,1), (0,5)\}$
	\end{center}
	(a) Following the definition of $W$, use the set-roster notation to enumerate all elements in $U$ and $V$. 
	\begin{itemize}
		\item for $U$:
		\\ 0+1 = 1, 4 < 1 < 6? No
		\\ 0+3 = 3, 4 < 3 < 6? No
		\\ 0+5 = 5, 4 < 5 < 6? Yes
		\\ 2+1 = 3, 4 < 3 < 6? No
		\\ 2+3 = 5,4 < 5 < 6? Yes
		\\ 2+5 = 7,4 < 7 < 6? No
		\item U = \{(0,5),(2,3)\}
		\item for $V$ = \{( 0,1)\}
	\end{itemize}
	(b) Indicate whether any of the relations $U$, $V$ , and $W$ are functions from $A$ to $B$. Justify your answers.
	\begin{itemize}
		\item U is a function since it has one $x$ value that maps to one $y$ value.
		\item V is not a function because it does not map out all the elements in $A$. 0 is mapped to $1$, but $2$ has no mapping
		\item W is not a function since 0 maps to two values, and a function can only map one x value corresponds to one y value.
	\end{itemize}
	

	
	\item Define a relation $T$ from $\mathbb{R}$ to $\mathbb{R}$ as follows: For all real numbers $x$ and $y, (x,y) \in T$, if and only if, they satisfy the equation $y^2 - 2x^2 =100$ is $T$ a function? Briefly explain your answer.
	\begin{itemize}
	\item We are given an equation $y^2 - 2x^2 =100$. In order to find out if $T$ is a function, x has to has to have only one corresponding y.
	\\ Let's solve for y in $y^2 - 2x^2 =100$: \\ 
	 $y^2 = -2x^2 + 100$ \\ 
	 $y = \pm \sqrt{-2x+100}$ \\ 
	 since we are taking the square root, we have \textit{two} possible $ x$ values corresponding a $y$ value.
	 \\ Therefore, $T$ does not follow the definition of  a function.
	\end{itemize}
\end{enumerate}
\end{document}