\documentclass{article}
\usepackage{amsmath}
\usepackage{amsfonts}
\usepackage{amssymb}

\title{HW 2}
\author{Anthony Kolotov}
\date{Jan 24 2025}

\begin{document}
	
	\maketitle
	
	\begin{enumerate}
		\item Write the following statements in symbolic form using the symbols \( \sim \), \( \lor \), and \( \land \) and the indicated letters to represent component statements. Let \( m \) = “More people are moving into Miami” and \( c \) = “the city gets more crowded.”
		\begin{enumerate}
			\item More people are moving into Miami but the city does get more crowded. \\
			Symbolic form: \( m \land c \)
			
			\item Neither more people are moving into Miami nor the city gets more crowded. \\
			Symbolic form: \( \sim m \land \sim c \)
		\end{enumerate}
		
		\item Write the following statements in symbolic form using the symbols \( \sim \), \( \lor \), and \( \land \) and the indicated letters to represent component statements. Let \( H \) = “John is healthy,” \( S \) = “John is strong,” and \( W \) = “John is wise.”
		\begin{enumerate}
			\item John is wise and healthy but not strong. \\
			Symbolic form: \( W \land H \land \sim S \)
			
			\item John is not wise but he is healthy and strong. \\
			Symbolic form: \( \sim W \land H \land S \)
			
			\item John is neither healthy, strong, nor wise. \\
			Symbolic form: \( \sim H \land \sim S \land \sim W \)
			
			\item John is neither strong nor wise, but he is healthy. \\
			Symbolic form: \( H \land \sim S \land \sim W \)
			
			\item John is wise, but he is not both healthy and strong. \\
			Symbolic form: \( W \land \sim (H \land S) \)
		\end{enumerate}
		
		\item Write truth tables for the following statement forms (make sure you follow the right order of precedence to parse the logic formula).
		\begin{enumerate}
			\item \( p \land \sim q \) \\
			
			\begin{tabular}{|c|c|c|c|}
				\hline
				$p$ & $q$ & $\sim q$ & $p \land \sim q$ \\
				\hline
				T & T & F & F \\
				T & F & T & T \\
				F & T & F & F \\
				F & F & T & F \\
				\hline
			\end{tabular}
			
			\item \( \sim (p \land q) \lor (p \lor q) \) \\
			
			\begin{tabular}{|c|c|c|c|c|c|}
				\hline
				$p$ & $q$ & $p \land q$ & $\sim (p \land q)$ & $p \lor q$ & $\sim (p \land q) \lor (p \lor q)$ \\
				\hline
				T & T & T & F & T & T \\
				T & F & F & T & T & T \\
				F & T & F & T & T & T \\
				F & F & F & T & F & T \\
				\hline
			\end{tabular}
			
			\item \( p \land (q \land r) \) \\
			
			\begin{tabular}{|c|c|c|c|c|}
				\hline
				$p$ & $q$ & $r$ & $q \land r$ & $p \land (q \land r)$ \\
				\hline
				T & T & T & T & T \\
				T & T & F & F & F \\
				T & F & T & F & F \\
				T & F & F & F & F \\
				F & T & T & T & F \\
				F & T & F & F & F \\
				F & F & T & F & F \\
				F & F & F & F & F \\
				\hline
			\end{tabular}
			
			\item \( p \land (\sim q \lor r) \) \\
			
			\begin{tabular}{|c|c|c|c|c|c|}
				\hline
				$p$ & $q$ & $r$ & $\sim q$ & $\sim q \lor r$ & $p \land (\sim q \lor r)$ \\
				\hline
				T & T & T & F & T & T \\
				T & T & F & F & F & F \\
				T & F & T & T & T & T \\
				T & F & F & T & T & T \\
				F & T & T & F & T & F \\
				F & T & F & F & F & F \\
				F & F & T & T & T & F \\
				F & F & F & T & T & F \\
				\hline
			\end{tabular}
		\end{enumerate}
		
		\item Use the truth table method to prove the following distributive laws.
		\begin{enumerate}
			\item \( p \land (q \lor r) = (p \land q) \lor (p \land r) \) \\
			
			\begin{tabular}{|c|c|c|c|c|c|c|c|}
				\hline
				$p$ & $q$ & $r$ & $q \lor r$ & $p \land (q \lor r)$ & $p \land q$ & $p \land r$ & $(p \land q) \lor (p \land r)$ \\
				\hline
				T & T & T & T & T & T & T & T \\
				T & T & F & T & T & T & F & T \\
				T & F & T & T & T & F & T & T \\
				T & F & F & F & F & F & F & F \\
				F & T & T & T & F & F & F & F \\
				F & T & F & T & F & F & F & F \\
				F & F & T & T & F & F & F & F \\
				F & F & F & F & F & F & F & F \\
				\hline
			\end{tabular}
			
			\item \( p \lor (q \land r) = (p \lor q) \land (p \lor r) \) \\
			
			\begin{tabular}{|c|c|c|c|c|c|c|c|}
				\hline
				$p$ & $q$ & $r$ & $q \land r$ & $p \lor (q \land r)$ & $p \lor q$ & $p \lor r$ & $(p \lor q) \land (p \lor r)$ \\
				\hline
				T & T & T & T & T & T & T & T \\
				T & T & F & F & T & T & T & T \\
				T & F & T & F & T & T & T & T \\
				T & F & F & F & T & T & T & T \\
				F & T & T & T & T & T & T & T \\
				F & T & F & F & F & T & F & F \\
				F & F & T & F & F & F & T & F \\
				F & F & F & F & F & F & F & F \\
				\hline
			\end{tabular}
		\end{enumerate}
		
		\item Assume that \( x \) is a particular real number and use De Morgan’s laws to write negations for the following statements.
		\begin{enumerate}
			\item \( x \geq -10 \) \\
			Negation: \( x < -10 \)
			
			\item \( -10 < x < 2 \) \\
			Negation: \( x \leq -10 \lor x \geq 2 \) \\ 
			
			\item \( x \leq -10 \) or \( x > 2 \) \\
			Negation: \( -10 < x \leq 2 \)
		\end{enumerate}
		
		\item Use the truth tables method to establish which of the following statement forms are tautologies, which are contradictions, and which are neither.
\begin{enumerate}
	\item \( (p \land q) \lor (\sim p \lor (p \land \sim q)) \) \\

	\begin{tabular}{|c|c|c|c|c|c|}
		\hline
		$p$ & $q$ & $\sim p$ & $p \land q$ & $p \land \sim q$ & $(p \land q) \lor (\sim p \lor (p \land \sim q))$ \\
		\hline
		T & T & F & T & F & T \\
		T & F & F & F & T & T \\
		F & T & T & F & F & T \\
		F & F & T & F & F & T \\
		\hline
	\end{tabular}
	\\
	This is a tautology
	
	\item \( ((\sim p \land q) \land (q \land r)) \land \sim q \) \\

	\begin{tabular}{|c|c|c|c|c|c|c|c|}
		\hline
		$p$ & $q$ & $r$ & $\sim p$ & $\sim q$ & $(\sim p \land q)$ & $(q \land r)$ & $((\sim p \land q) \land (q \land r)) \land \sim q$ \\
		\hline
		T & T & T & F & F & F & T & F \\
		T & T & F & F & F & F & F & F \\
		T & F & T & F & T & F & F & F \\
		T & F & F & F & T & F & F & F \\
		F & T & T & T & F & T & T & F \\
		F & T & F & T & F & T & F & F \\
		F & F & T & T & T & F & F & F \\
		F & F & F & T & T & F & F & F \\
		\hline
	\end{tabular}
	\\
	This is a contradiction
	
	\item \( (\sim p \lor q) \lor (p \land \sim q) \) \\
	\begin{tabular}{|c|c|c|c|c|c|c|}
		\hline
		$p$ & $q$ & $\sim p$ & $\sim q$ & $\sim p \lor q$ & $p \land \sim q$ & $(\sim p \lor q) \lor (p \land \sim q)$ \\
		\hline
		T & T & F & F & T & F & T \\
		T & F & F & T & F & T & T \\
		F & T & T & F & T & F & T \\
		F & F & T & T & T & F & T \\
		\hline
	\end{tabular}
	\\
	This is a tautology 
	
	\item \( (p \rightarrow r) \leftrightarrow (q \rightarrow r) \) \\
	\begin{tabular}{|c|c|c|c|c|c|c|c|}
		\hline
		$p$ & $q$ & $r$ & $p \rightarrow r$ & $q \rightarrow r$ & $(p \rightarrow r) \leftrightarrow (q \rightarrow r)$ \\
		\hline
		T & T & T & T & T & T \\
		T & T & F & F & F & T \\
		T & F & T & T & T & T \\
		T & F & F & F & T & F \\
		F & T & T & T & T & T \\
		F & T & F & T & F & F \\
		F & F & T & T & T & T \\
		F & F & F & T & T & T \\
		\hline
	\end{tabular}
	\\
	This is neither
	
	
	\item \( (p \rightarrow (q \rightarrow r)) \leftrightarrow ((p \land q) \rightarrow r) \) \\
	\begin{tabular}{|c|c|c|c|c|c|c|c|c|c|}
		\hline
		$p$ & $q$ & $r$ & $q \rightarrow r$ & $p \rightarrow (q \rightarrow r)$ & $p \land q$ & $(p \land q) \rightarrow r$ & $(p \rightarrow (q \rightarrow r)) \leftrightarrow ((p \land q) \rightarrow r)$ \\
		\hline
		T & T & T & T & T & T & T & T \\
		T & T & F & F & F & T & F & F \\
		T & F & T & T & T & F & T & T \\
		T & F & F & F & F & F & F & T \\
		F & T & T & T & T & F & T & T \\
		F & T & F & F & T & F & F & T \\
		F & F & T & T & T & F & T & T \\
		F & F & F & F & T & F & T & T \\
		\hline
	\end{tabular}
	\\
	This is a neither
\end{enumerate}
		
		\item Write each of the following three statements in symbolic form and determine which pairs are logically equivalent. Make sure to include truth tables and a brief explanation.
		\begin{enumerate}
			\item If it walks like a duck and it talks like a duck, then it is a duck. \\
			Symbolic form: \( (P \land Q) \rightarrow D \)
			
			\item Either it does not walk like a duck or it does not talk like a duck, or it is a duck. \\
			Symbolic form: \( (\sim P \lor \sim Q) \lor D \)
			
			\item If it does not walk like a duck and it does not talk like a duck, then it is not a duck. \\
			Symbolic form: \( (\sim P \land \sim Q) \rightarrow \sim D \)
		\end{enumerate}
		\textbf{Equivalence:} The first and third statements are logically equivalent because they both express the condition for a duck to exist based on walking and talking. The second statement is a different logical form but equivalent to the first in certain logical contexts.
		
		\item Use the logical equivalence \( p \rightarrow q \equiv \sim p \lor q \) and de Morgan’s laws to rewrite the following statement forms using \( \land \) and \( \sim \) only (that is, you should eliminate all \( \lor \), \( \rightarrow \) and \( \leftrightarrow \) symbols in your answer statement forms).
		\begin{enumerate}
			\item \( p \land \sim q \rightarrow r \) \\
			Rewritten form: \( \sim (p \land \sim q) \lor r \equiv \sim p \lor q \lor r \)
			
			\item \( p \lor \sim q \rightarrow r \lor q \) \\
			Rewritten form: \( \sim p \lor q \lor r \lor q \equiv \sim p \lor r \)
			
			\item \( (p \rightarrow (q \rightarrow r)) \leftrightarrow ((p \land q) \rightarrow r) \) \\
			Rewritten form: \( \sim p \lor (q \lor r) \leftrightarrow \sim p \land q \lor r \)
		\end{enumerate}
		
		\item Rewrite the following statements which use “necessary condition” or “sufficient condition” form and turn them into statements using “if-then” form.
		\begin{enumerate}
			\item A necessary condition for Jon’s team to win the championship is that it wins the rest of its games. \\
			\begin{enumerate}
				\item  If Jon’s team wins the rest of its games, then it will be a necessary condition for winning the championship.
			\end{enumerate} 
			\item Winning this championship is a necessary condition for Andy to qualify for the Paris 2024 Olympics Games.
				\begin{enumerate}
					\item If Andy wins this championship, then he will quality for the Paris 2024 Olympics games.
				\end{enumerate}
		\end{enumerate}
	\end{enumerate}
	
\end{document}
