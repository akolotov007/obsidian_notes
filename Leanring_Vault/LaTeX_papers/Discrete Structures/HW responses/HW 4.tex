\documentclass{article}
\usepackage{amsmath, amssymb}

\title{HW 4}
\date{\today}
\author{Anthony Kolotov}

\begin{document}
	
	\maketitle
	

	
	\begin{enumerate}
		\item Find the first five terms of the sequence defined by each of these recurrence relations and initial conditions:
		\begin{enumerate}
			\item $a_n = 4a_{n-1} + 1, \quad a_0 = 1$;
			\subitem $a_1=5, a_2 = 21, a_3 = 85, a_4 = 341, a_5 = 1365$
			\item $a_n = a_{n-1}^2 - 2, \quad a_0 = 2$;
			\subitem $a_{1\rightarrow 5}=2$
			\item $a_n = a_{n-1}^2 - 1, \quad a_1 = 2$;
			\subitem $ a_2=3, a_3 = 8, a_4 = 63, a_5 = 3968 $
			\item $a_n = a_{n-1} + 3a_{n-2}, \quad a_0 = 1, \quad a_1 = 2$;
			\subitem $ a_2 = 5, a_3 = 11, a_4 = 26, a_5 = 59$
			\item $a_n = n a_{n-1} + n^2 a_{n-2}, \quad a_0 = a_1 = 1$.
			\subitem $ a_2 = 6 , a_3 = 27, a_4 = 204, a_5 = 1695$
		\end{enumerate}
		
		\item Let $x > 0$ be a positive real number. What are the possible values of $\lfloor 3x \rfloor$ in terms of $\lfloor x \rfloor$? Under what conditions will $\lfloor 3x \rfloor$ take each of these values? Give one example for each case.
		
		\subitem $\lfloor x \rfloor$ would be rounded down to the closest integer. For example: $\frac{3}{2} = 1.5, \lfloor 1.5 \rfloor = 1.$ Multiplying the X value by 3 and then getting the floor of it would still be rounded down to the closest integer. Following the previous example: $ \lfloor3(1.5)\rfloor = \lfloor4.5\rfloor = 4$ 
		
		
		\item The formula
		\[
		1 + r + r^2 + \dots + r^n = \frac{r^{n+1} - 1}{r - 1}
		\]
		holds for every integer $n \geq 0$ and every real number $r \neq 1$. Use this formula to find the sum of each of the following sequences:
		\begin{enumerate}
			\item $1 + 2 + 2^2 + \dots + 2^{n-1}$
				\subitem $S_n = 2^n - 1$
			\item $3^{n-1} + 3^{n-2} + \dots + 3 + 1$
				\subitem $S_n = \frac{3^{n}-1}{2} $
			\item $2^n + 3 \cdot 2^{n-2} + \dots + 3 \cdot 2 + 3$
				\subitem $3\cdot(2^{n}+1)$
			\item $2^n - 2^{n-1} + 2^{n-2} - \dots + (-1)^{n-1} \cdot 2 + (-1)^n$
				\subitem $S_n \frac{2^{n+1}+(-1)^n}{3}$
		\end{enumerate}
		
		\item Calculate the number of trailing zeros of $5678!$.
			\subitem we need to count how many factors of 5 there are in 5678!
		\begin{itemize}
		\item 5678/5 = 1135 
		\item 5678/25 = 227  
		\item 5678/125 = 45  
		\item 5678/625 = 9  
		\item 5678/3125 = 1
		\item total = 1417 zeros
		\end{itemize}
		
		
		\item Deduce the formula for the summation of cubes:
		\[ \sum_{i=1}^{n} i^3 = \frac{n^2 (n+1)^2}{4} = \left( \sum_{i=1}^{n} i \right)^2. \]
		\subitem Starting from the RHS (right hand side)
		$(\sum_{i=1}^{n} i)^2$ \\
		$(1+2+3+\cdots + n) \cdot (1+2+3+\cdots + n)$ \\ 
		$1(1)+ 2(1+2+1) 3 (1+2+3+2+1)+ \cdots n[1+2+\cdots + n+(n-1) + \cdots + 1]$ factoring out  \\ 
		$1(1) + 2(2+2) + 3(3+3+3) + \cdots + n(n+n \cdots + n) = $ adding up \\ 
		$1(1) + 2(2\cdot2) + 3(3\cdot3) + \cdots + n(n\cdot n)= $ re-writing expression \\ 
		= $1^3 + 2^3 + 3^3 +\cdots + n^3$ = LHS (left hand side)
		
		
		
		\item Evaluate the following products:
		\begin{enumerate}
			\item $\prod_{i=0}^{50} i^2$
			\subitem $0^2 \cdot 1^2 \cdots 50^2 = 0$ (since first term is 0)
			\item $\prod_{i=5}^{8} i$
			\subitem $5 \times 6 \times 7 \times 8 = 1,680 $
			\item $\prod_{i=1}^{10} 2$
			\subitem $2^10 = 1024$ (since 2 is the constant, we count count up to 10)
			\item $\prod_{i=1}^{100} (-1)^i$
			\subitem $-1^{100} = 1$ (since 100 is even, there are an even number of -1 factors)
		\end{enumerate}
		
		\item Evaluate the following sum and product:
		\begin{enumerate}
			\item $\prod_{i=1}^{n} \frac{i}{i+1}$
			\subitem $1/2 \cdot 2/3 \cdot 3/4 \cdot 4/5 \cdots \frac{i}{i+1} $ it is a telescoping pattern... = $\frac{1}{n+1}$
			
			\item $\sum_{i=1}^{n} \frac{1}{i(i+1)}$ \\ 
			 $ \frac{1}{i} - \frac{1}{i+1}  = \frac{(k+1)-k}{k(k+1)}=\frac{1}{i(i+1)} $ \\
			 
			 $(1 - 1/2)+ (1/2 - 1/3) +  (1/3 - 1/4) \cdots + (\frac{1}{n-1} - \frac{1}{n}) + (\frac{1}{n} - \frac{1}{n+1})$
			  becomes a telescoping pattern, leaving us with: \\ 
			 $1 - \frac{1}{n+1}$ Example 5.1.10 in the book
			 
		\end{enumerate}
		
		\item Show that if $a \mid b$ and $b \mid a$, where $a$ and $b$ are integers, then either $a = b$ or $a = -b$.
		
		given $a|b$, $b = ka$ for some integer $k$ \\ 
		given $b|a$, $a = bm$ for some integer $m$ \\
		substituting $b=ka$ into $a=bm$ :\\ 
		$a = m(ka) = (mk)a = a$  since a $\not= 0$ we can divide $a$ out:\\
		$mk=1$ \\ 
		since  $m$ and $k$ are integers, the only solutions are: \\
		$m=k=1$ or $m=k=-1$\\
		Thus $b=a$ or $b=-a$
		
		\item Prove that if $n$ is an integer not divisible by $3$, then $n^2 - 1$ is divisible by $3$.
		
		
		\item Prove that if $n$ is a positive integer, then $4$ does not divide $n^2 + 2$. (Hint: Consider cases when $n$ is even and odd separately.)
		 \\ \\ 
		case 1: n is even, so $n = 2k$, substituting that: \\ 
		$(2k)^2 + 2 = 4k^2 +2$ \\ 
		Since $4k^2$ can be divided by 4, we analyze the term: $4k^2 +2 \equiv 2$ (mod 4) \\ Since 2 $\not=0$ mod 4, we conclude that 4  does not divide $n^2 + 2$ when $n$ is even
		\\ \\
		case 2: when n is odd, so $n=2k+1 $ for some integer $k$: \\
		$(2k+1)^2 +2  = 4k^2 +4k + 1 + 2 = 4k^2 + 4k +3$ \\
		Since $4k^2$ is divisible by 4, we look at the remainder: \\
		$4k^2 + 4k + 3 \equiv 3$ (mod 4) \\ 
		since 3 $\not=$ 0 (mod 4), we can conclude that 4 does not divide $n^2 + 2$ when $n$ is odd

		
	\end{enumerate}
	
\end{document}
